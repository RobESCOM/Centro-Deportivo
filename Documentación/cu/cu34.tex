% \IUref{IUAdmPS}{Administrar Planta de Selección}
% \IUref{IUModPS}{Modificar Planta de Selección}
% \IUref{IUEliPS}{Eliminar Planta de Selección}


% Copie este bloque por cada caso de uso:
%-------------------------------------- COMIENZA descripción del caso de uso.


%\begin{UseCase}[archivo de imágen]{UCX}{Nombre del Caso de uso}{
	\begin{UseCase}{CU34}{Suspender Servicios.}{
		Permite al gerente de sucursal, recepcionista, suspender un servicio debido a fechas de expiración, cambios de servicios.
	Con el fin de no hacer una eliminacion total si no parcial con el objetivo de activar nuevamente el servicio en el sistema.
	}
		\UCitem{Versión}{1.0}
		\UCitem{Actor}{Gerente de Sucursal, Recepcionista}
		\UCitem{Propósito}{Dar de baja un servicio del sistema, con el objetivo de que no se cuente su participación dentro del sistema y de la sucursal, sin embargo el servicio seguira existiendo pero con un estado de inactividad.}
		\UCitem{Entradas}{Estado del servicio}
		\UCitem{Origen}{Teclado.}
		\UCitem{Salidas}{Se muestra un mensaje de "servicio actualizado".}
		\UCitem{Destino}{El dato ya no se mostrará en el área de consultas, mas no será eliminado totalmente del sistema.}
		\UCitem{Precondiciones}{
Para suspender un servicio se requiere que esté registrado previamente.
		Se requiere que el servicio no se haya eliminado del sistema.}
		\UCitem{Postcondiciones}{El actor puede cambiar el estado del servicio nuevamente.}
		\UCitem{Errores}{
No se cuenta con ningun dato registrado.
		No se efectua correctamente la operacion.}
		\UCitem{Tipo}{Caso de uso primario.}
		\UCitem{Observaciones}{}
		\UCitem{Autor}{Francisco Garcia Enriquez}
		\UCitem{Revisor}{Roberto mendoza Saavedra}
	\end{UseCase}

	\begin{UCtrayectoria}{Principal}
		\UCpaso Solicita ingreso al sistema.
		\UCpaso Toma la sesion del actor
		\UCpaso Mustra en la página principal el menú de opciones.
		\UCpaso[\UCactor] Selecciona del menú la opción servicios.
		\UCpaso Muestra las opciones que el actor puede realizar: Agregar servicio, consultar servicios, eliminar servicios, suspender servicios y actualizar servicios.
		\UCpaso[\UCactor] Selecciona la opción, Actualizar servicios.
		\UCpaso Carga una tabla con los servicios registrados, adjunto de un boton de edición [Trayectoria A].
		\UCpaso[\UCactor] Presiona el boton de editar.
		\UCpaso Muestra una página deonde se cargan los datos del servicio en los campos del formulario.
		\UCpaso[\UCactor] Cambia el estado del servicio seleccionado a inactivo, en el "campo estado del servicio".
		\UCpaso[\UCactor] Confirma el cambio presionando el boton de actualizar.
		\UCpaso Actualza el estado del servicio y muestra un mensaje MSG.- "Los datos se actualizaron correctamente". \Trayref{A}
		\UCpaso[\UCactor] Puede consultar el estado del servicio en la sección de consultas.
		\UCpaso[\UCactor] Finaliza su operacion dentro del área y puede regresar al menú de opciones de la sección servicios.
	\end{UCtrayectoria}

		\begin{UCtrayectoriaA}{A}{No existe ningun dato registrado.}
			\UCpaso muestra el mensaje MS7- "No existe ningun registro".
			\UCpaso[\UCactor] Finaliza su operación dentro del área de consulta.
			\UCpaso[\UCactor] puede regresar al menú de opciones de la sección, mediante el boton de "regresar al menu de inició".
		\end{UCtrayectoriaA}
	
		%-------------------------------------- TERMINA descripción del caso de uso.