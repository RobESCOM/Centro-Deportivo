% \IUref{IUAdmPS}{Administrar Planta de Selección}
% \IUref{IUModPS}{Modificar Planta de Selección}
% \IUref{IUEliPS}{Eliminar Planta de Selección}


% Copie este bloque por cada caso de uso:
%-------------------------------------- COMIENZA descripción del caso de uso.

%\begin{UseCase}[archivo de imágen]{UCX}{Nombre del Caso de uso}{
	\begin{UseCase}{CU6.0}{Consultar Áreas Registradas}{
		Para esta sección el gerente de sucursal podrá consultar las áreas que se encuentran disponibles.
	}
		\UCitem{Versión}{1.0}
		\UCitem{Actor}{Gerente de sucursal}
		\UCitem{Propósito}{Que el Gerente de sucursal pueda visualuzar las áreas que se encuentran disponibles.}
		\UCitem{Entradas}{El nombre del área mediante una lista desplegable. \IUref{IU8}{Pantalla de consulta de areas.}}
		\UCitem{Origen}{Los datos serán mostrados por medio de una tabla.}
		\UCitem{Salidas}{Se mostrará el nombre del área con todos sus atributos tales como:
			      Nombre,tipo de material, ancho, largo, capacidad,responsable.}
		\UCitem{Destino}{Los datos serán mostrados en la pantalla.}
		\UCitem{Precondiciones}{Para visualizar las áreas se requiere, que el área se encuentre registrada previamente, en caso de haber áreas registradas no se mostrará dato alguno en la tabla, pero se podrá accesar al área de consulta.}
		\UCitem{Postcondiciones}{El actor podrá regresar al menú de opciones para realizar otras operaciones en el sistema.}
		\UCitem{Errores}{ {\bf E7:} ``No se tiene ningún registro.'' -- El sistema muestra el Mensaje {\bf MSG1-}``No existe ningun registro.'' y el actor puede regresar al menu principal.}
		\UCitem{Tipo}{Caso de uso primario}
		\UCitem{Observaciones}{}
		\UCitem{Autor}{Francisco García Enríquez.}
		\UCitem{Revisor}{Martin Carrillo.}
	\end{UseCase}

	\begin{UCtrayectoria}{Principal}
		\UCpaso[\UCactor] Solicita el ingreso al apartado de clientes seleccionando la opción ``Clientes'' de la \IUref{IU6}{Pantalla de perfil de empleado}.
		\UCpaso Toma la sesión de gerente de sucursal.
		\UCpaso Muestra en la página principal las opciones que tiene disponibles para realizar operaciones el actor. \IUref{IU8}{Pantalla de menu de opciones inicial.}
		\UCpaso[\UCactor] Selecciona del menú la opción Áreas.
		\UCpaso Muestra las opciones que el gerente pueda realizar: Registrar Áreas, Consultar Áreas, Eliminar Áreas, Dar de Baja Áreas y Actualizar Áreas. En la sección \IUref{IU18}{Pantalla de menu de opciones áreas}.
		\UCpaso[\UCactor] Selecciona la opción de Consultar Áreas.
		\UCpaso Mostrará los datos de las áreas registradas mediante una tabla  \Trayref{A}. \IUref{IU8}{Pantalla de consulta de áreas.}	
		\UCpaso[\UCactor] Podrá visualizar los datos de las áreas registradas.
		\UCpaso[\UCactor] Finaliza su operación dentro del área de consulta.
		\UCpaso[\UCactor] Regresará al menú de opciones de la sección \IUref{IU18}{Pantalla de menu de opciones áreas} mediante el botón de menú de inicio.				
	\end{UCtrayectoria}

\begin{UCtrayectoriaA}{A}{No existe ningun dato registrado.}
			\UCpaso[\UCactor] Muestra el mensaje {\bf MS7-}``No existe ningun registro''
			\UCpaso[\UCactor] Finaliza su operación dentro del área de consulta. 
			\UCpaso[\UCactor] Puede regresar al menú de opciones de la sección \IUref{IU18}{Pantalla de menu de opciones áreas} mediante el botón de menú de inicio.
		\end{UCtrayectoriaA}