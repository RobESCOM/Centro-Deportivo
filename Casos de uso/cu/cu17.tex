% \IUref{IUAdmPS}{Administrar Planta de Selección}
% \IUref{IUModPS}{Modificar Planta de Selección}
% \IUref{IUEliPS}{Eliminar Planta de Selección}

% 


% Copie este bloque por cada caso de uso:
%-------------------------------------- COMIENZA descripción del caso de uso.

%\begin{UseCase}[archivo de imágen]{UCX}{Nombre del Caso de uso}{
	\begin{UseCase}{CU11}{Registrar en el sistema una nueva sucursal.}{
		Almacenar los datos referentes a una sucursal próxima a inaugurarse.
	}
		\UCitem{Versión}{0.1}
		\UCitem{Actor}{Gerente de operación de negocio}
		\UCitem{Propósito}{Registrar una nueva sucursal en el sistema tiene como propósito ampliar las posibilidades de los clientes para asistir a las áreas y cursos que tenga la sucursal.}
		\UCitem{Entradas}{Los datos de entrada son el nombre de la sucursal, dirección, código postal, estado en el que se encuentra ubicada la sucursal, teléfonos, correo electrónico, datos referentes al gerente, áreas con las que cuenta la sucursal, servicios con los que cuenta la sucursal. }
		\UCitem{Origen}{Los datos de entrada tienen su origen a través del teclado del computador del actor.}
		\UCitem{Salidas}{Se mostrará un mensaje de registro exitoso en el caso de que no haya errores, si los hay se mostrará un mensaje que informe de dicho error.}
		\UCitem{Destino}{Los mensajes a mostrar se desplegaran en la pantalla del computador del actor. Los datos se enviarán al servidor para procesarlos.}
		\UCitem{Precondiciones}{El actor ingreso al sistema mediante un login. El sistema se encuentra en la pantalla de formulario para dar de alta una nueva sucursal, en el cual el actor llenará todos los campos que el formulario muestre.}
		\UCitem{Postcondiciones}{En la base de datos del sistema se tendra un nuevo registro de una sucursal, de las áreas con la que ésta cuenta, de los servicios que ofrece la sucursal y los datos del gerente de la sucursal.}
		\UCitem{Errores}{ 1. Los datos proporcionados por el actor no son del tipo específicado en los campos del formulario.

2. La sucursal ya se encuentra registrada.

3. La persona que se específica con el cargo de gerente de sucursal ya se encuentra como gerente de otra sucursal.

4. Es responsabilidad del actor añadir las áreas y servicios con los que realmente cuente la sucursal próxima a inaugurarse. }
		\UCitem{Tipo}{Caso de uso primario}
		\UCitem{Observaciones}{}
		\UCitem{Autor}{Fernández Quiñones Isaac.}
		\UCitem{Revisó}{}
	\end{UseCase}

	\begin{UCtrayectoria}{Principal}
		\UCpaso[\UCactor] Ingresa a la plataforma web y proporciona su userName y password mediante la \IUref{IU23}{Pantalla de Control de Acceso}\label{CU11Login} para acceder al sistema.
		\UCpaso Válida que el actor se encuentre dado de alta en el sistema. Se utiliza la regla \BRref{BR117}{Determinar si el usuario tiene acceso al sistema.} \Trayref{A}.
		\UCpaso Despliega la \IUref{IU32}{Pantalla de formulario para registro de nueva sucursal} que contiene los campos necesarios para registrar en el sistema la nueva sucursal.
		\UCpaso[\UCactor] Introduce el nombre de la nueva sucursal.
		\UCpaso[\UCactor] Selecciona la fecha de inauguración de la sucursal.
		\UCpaso[\UCactor] Selecciona el estado en el que se encontrará ubicada la nueva sucursal.
		\UCpaso[\UCactor] Escribe el número teléfonico que será asociado a la sucursal. \Trayref{B}\label{CU11AgregarTelefono}.
		\UCpaso[\UCactor] Escribe el correo electrónico que será asignado a la sucursal. \Trayref{C}\label{CU11AgregarMail}.
		\UCpaso[\UCactor] Introduce la calle donde se encuentra ubicada la sucursal.
		\UCpaso[\UCactor] Proporciona el nombre de la colonia donde esta la sucursal.
		\UCpaso[\UCactor] Escribe el nombre de la delegación a la que pertenece la sucursal.
		\UCpaso[\UCactor] Proporciona el código postal de la sucursal.
		\UCpaso[\UCactor] Da click sobre el triangulo que apunta a la pregunta ¿Con qué áreas cuenta la sucursal?
		\UCpaso	Muestra un listado de los nombres de áreas que se encuentran almacenados en la base de datos.
		\UCpaso[\UCactor] Selecciona las áreas con las que cuenta la nueva sucursal.
		\UCpaso[\UCactor] Da click sobre el triangulo que apunta a la pregunta ¿Con qué servicios cuenta la sucursal?
		\UCpaso	Muestra un listado de los nombres de los servicios que se encuentran almacenados en la base de datos.
		\UCpaso[\UCactor] Selecciona los servicios con los que contará la nueva sucursal.		
		\UCpaso[\UCactor] Proporciona el nombre del gerente de la nueva sucursal.
		\UCpaso[\UCactor] Proporciona el primer y segundo apellido del gerente de la nueva sucursal.
		\UCpaso[\UCactor] Introduce la direccion del gerente de la nueva sucursal.
		\UCpaso[\UCactor] Llena el campo descripción. Esta es información de la sucursal.
		\UCpaso[\UCactor] Presiona el botón \IUbutton{Envíar}. \label{CU11EnviarFormulario}.
		\UCpaso verifica los datos introducidos por el \UCactor. \BRref{BR118}{Determinar si los datos de los campos de un formulario son del tipo adecuado} \Trayref{D}.
		\UCpaso Almacena los datos en la base de datos.
		\UCpaso Muestra el \IUref{UI88}{Mensaje de registro exitoso}. 
	\end{UCtrayectoria}
		
		\begin{UCtrayectoriaA}{A}{El actor no cuenta con las credenciales válidas para poder ingresar al sistema.}
			\UCpaso Muestra el mensaje {\bf MSG1-}``Usuario [{\em y/o}] contraseñas no validos.''.
			\UCpaso[\UCactor] Oprimé el botón \IUbutton{Aceptar}.
			\UCpaso Continua en el paso \ref{CU11Login} del \UCref{CU11}.
		\end{UCtrayectoriaA}
		
		\begin{UCtrayectoriaA}{B}{Se desea ingresar más de un teléfono}
			\UCpaso[\UCactor] Presionó el botón "más" de la interfaz de usuario \IUref{IU34}{Añadir otro campo}, para agregar un número teléfonico distinto.
			\UCpaso Muestra un campo de texto adicional para el llenado del mismo con un número teléfonico.
			\UCpaso Continua en el paso \ref{CU11AgregarTelefono} del \UCref{CU11}. 
		\end{UCtrayectoriaA}
		
		\begin{UCtrayectoriaA}{C}{Se desea agregar un correo electrónico}
			\UCpaso[\UCactor] Presionó el botón "más" de la interfaz de usuario \IUref{IU35}{Añadir otro email}.
			\UCpaso Muestra un campo de texto adicional para el llenado del mismo con un correo electrónico.
			\UCpaso Continua en el paso \ref{CU11AgregarMail} del \UCref{CU11}. 
		\end{UCtrayectoriaA}		
		
		\begin{UCtrayectoriaA}{D}{Alguno de los campos no se específico o los datos no concuerdan con el tipo de dato esperado.}
			\UCpaso Muestra el mensaje {\bf MSG1-}``Uno o más [{\em campos}] no tienen el formato adecuado''.
			\UCpaso[\UCactor] Oprime el botón \IUbutton{Aceptar}.
			\UCpaso Pone el foco en el campo donde se encontro el primer error y marca con un color el borde de los demás campos que tengan un tipo de dato no esperado por el campo, para que el usuario pueda identificar cuales campos necesita corregir. 
			\UCpaso[\UCactor] Corrige el valor erroneo, del campo que tiene el foco y los demas campos con valor erroneo, a un valor correcto.
			\UCpaso Continua en el paso \ref{CU11EnviarFormulario} del \UCref{CU11}.
		\end{UCtrayectoriaA}
		
%-------------------------------------- TERMINA descripción del caso de uso.
%%%%%%%%%%%%%%%%%%%%%%%%%%%%%%%%%%%%%%

%-------------------------------------- COMIENZA caso de uso para actualizar datos
\begin{UseCase}{CU12}{Actualizar los datos de una sucursal,}{
		Modificar los datos del registro de una sucursal que se encuentran almacenado en la base de datos del sistema.
	}
		\UCitem{Versión}{0.1}
		\UCitem{Actor}{Gerente de operación de negocio.}
		\UCitem{Propósito}{Los datos de una sucursal, tales como los gerente y personal e incluso la direccion de una sucursal, no son permanentes y estos tienden a cambiar. Es por esta razon que se crea el caso de uso \UCref{CU2} para poder actualizar los datos de las sucursales.}
		\UCitem{Entradas}{Los nuevos datos para actualizar un registro son seleccionados por el actor.}
		\UCitem{Origen}{El teclado del equipo de computo del actor.}
		\UCitem{Salidas}{Mensaje de que la modificacion de los datos del regitro de la sucursal seleccionados fue exitosa. Mensaje de error en el caso de que no se llene de manera correcta un campo.}
		\UCitem{Destino}{La pantalla del equipo de computo del actor.}
		\UCitem{Precondiciones}{1. El actor debio haber ingresado al sistema.
		
2. Debe existir el registro a modificar.}
		\UCitem{Postcondiciones}{Se tendra una actualizacion de los datos del registro en la base de datos del sistema.}
		\UCitem{Errores}{}
		\UCitem{Tipo}{Caso de uso primario}
		\UCitem{Observaciones}{}
		\UCitem{Autor}{Fernández Quiñones Isaac.}
	\end{UseCase}

	\begin{UCtrayectoria}{Principal}
		\UCpaso[\UCactor] Entra a la plataforma en linea. Mediante la \IUref{IU23}{Pantalla de Control de Acceso}\label{CU17Login} para entrar en el sistema.
		\UCpaso[\UCactor] proporciona sus credenciales para ingresar al sistema.
		\UCpaso Válida que el actor se encuentre dado de alta en el sistema. Se utiliza la regla \BRref{BR117}{Determinar si el usuario tiene acceso al sistema.} \Trayref{A}.
		\UCpaso Despliega la \IUref{IU32}{Pantalla de sucursales registradas}.
		\UCpaso[\UCactor] selecciona la sucursal a actualizar datos. \label{CU17SeleccionarSeminario}.
		\UCpaso Muestra los datos de la sucursal en un formulario. Este formulario contiene los datos previos a la modificación.
		\UCpaso[\UCactor] modifica los campos necesarios y presiona el boton \IUbutton{Actualizar}. 
		\UCpaso Verifica los datos proporcionados por el \UCactor. \BRref{BR118}{Determinar si los datos de los campos de un formulario son del tipo adecuado} \Trayref{B}.
		\UCpaso Almacena los cambios en la base de datos.
		\UCpaso Muestra el \IUref{UI88}{Datos actualizados satisfactoriamente}. 
		\UCpaso[\UCactor] Presiona el boton \IUbutton{Aceptar}. 
		\UCpaso Pregunta si quiere modificar algun otro registro de una sucursal. \Trayref{C} \Trayref{D}
	\end{UCtrayectoria}
	
		\begin{UCtrayectoriaA}{A}{El actor no cuenta con las credenciales validas para poder ingresar al sistema.}
			\UCpaso Muestra el Mensaje {\bf MSG1-}``Usuario [{\em y/o}] contraseña no validos.''.
			\UCpaso[\UCactor] Oprime el botón \IUbutton{Aceptar}.
			\UCpaso Continua en el paso 1 del \UCref{CU2}
		\end{UCtrayectoriaA}
		
		\begin{UCtrayectoriaA}{B}{Alguno de los campos no se especifico o los datos no concuerdan con el tipo esperado.}
			\UCpaso Muestra el Mensaje {\bf MSG1-}``Un [{\em campo}] no se lleno correctamente.''.
			\UCpaso[\UCactor] Oprime el botón \IUbutton{Aceptar}.
			\UCpaso Pone el foco en el campo donde se encontro el primer error y marca los demas campos con algun error en valor proporcionado por el usuario.
			\UCpaso[\UCactor] Corrige el valor erroneo, del campo que tiene el foco y los demas campos con valor erroneo, a un valor correcto.
			\UCpaso Continua en el paso 5 del caso de uso \UCref{CU2}.
		\end{UCtrayectoriaA}
 
		\begin{UCtrayectoriaA}{C}{}
			\UCpaso[\UCactor] Presiona el boton \IUbutton{Si}.
			\UCpaso Continua en el paso 4 del caso de uso \UCref{CU2}.
		\end{UCtrayectoriaA}
		
		\begin{UCtrayectoriaA}{D}{}
			\UCpaso[\UCactor] Presiona el boton \IUbutton{No}.
			\UCpaso Continua en el paso 12 del caso de uso \UCref{CU2}.
		\end{UCtrayectoriaA}
%------------------------------------- TERMINA caso de uso para actualizar datos de sucursal

%------------------------------------- COMIENZA caso de uso para dar de baja sucursal
\begin{UseCase}{CU119}{Baja de sucursal.}{
		En el caso de clausura o cierre temporal de  una sucursal se dara de baja la sucursal pero no se eliminara el registro de la base de datos.
	}
		\UCitem{Versión}{0.1}
		\UCitem{Actor}{Gerente de operación de negocio}
		\UCitem{Propósito}{Quitar de las tablas que se muestran en la pantalla de los computadores de los clientes, gerentes, intructores, etc. el registro de la sucursal para que ningun usuario trate de acceder a registrar un curso y advertir a los usuarios que la sucursal no esta disponible.}
		\UCitem{Entradas}{La fecha de cierre temporal o clausura de la sucursal, ademas de una descripcion explicando el porque se origina el cierre temporal o clausura.}
		\UCitem{Origen}{El teclado del computador del actor.}
		\UCitem{Salidas}{Se mostrará el mensaje {\bf MSG3-}``La [{\em sucursal}] fue dada de baja.''.}
		\UCitem{Destino}{La pantalla del equipo de cómputo del actor.}
		\UCitem{Precondiciones}{La sucursal no debe estar dada de baja.}
		\UCitem{Postcondiciones}{El sistema tendra una sucursal mas dada de baja. No se mostrara mas este registro a los usuarios, tales como el gerente de sucursal, clientes, instructores, etc. Para el actor gerente de operaciones de negocio si estaran disponibles las sucursales dadas de baja.}
		\UCitem{Errores}{1. La sucursal no se puede dar de baja.

2. No se lleno el campo de descripcion del cierre temporal o clausura.

3. La fecha ingresada es pasada con respecto la fecha en el que se  intenta dar de baja la sucursal en el sistema.}
		\UCitem{Tipo}{Caso de uso primario}
		\UCitem{Observaciones}{La fecha ingresada debe ser la actual o no mayor a un mes, o se puede poner cualquier fecha que no sea pasada a la fecha del registro de la sucursal.}
		\UCitem{Autor}{Fernández Quiñones Isaac.}
	\end{UseCase}

	\begin{UCtrayectoria}{Principal}
		\UCpaso[\UCactor] Ingresa a la plataforma web e ingresa su usuario y contraseña mediante la \IUref{IU23}{Pantalla de Control de Acceso}\label{CU17Login} para entrar en el sistema.
		\UCpaso Válida que el actor se encuentre dado de alta en el sistema. Se utiliza la regla \BRref{BR117}{Determinar si el usuario tiene acceso al sistema.} \Trayref{A}.
		\UCpaso Despliega la \IUref{IU99}{Pantalla dar de baja sucursales} con los campos necesarios para registrar en el sistema la nueva sucursal.
		\UCpaso[\UCactor] Llena todos los campos del formulario y envia el formulario. \Trayref{B}\label{CU17SeleccionarSeminario}.
		\UCpaso verifica los datos introducidos por el \UCactor. \BRref{BR118}{Determinar si los datos de los campos de un formulario son del tipo adecuado} \Trayref{C}.
		\UCpaso Almacena los datos en la base de datos.
		\UCpaso Muestra el \IUref{UI88}{Mensaje de registro exitoso}. 
		\UCpaso Pregunta al estudiante si desea imprimir un comprobante de la inscripción.		
	\end{UCtrayectoria}
		
		\begin{UCtrayectoriaA}{A}{El Estudiante no puede inscribir un Seminario}
			\UCpaso Muestra el Mensaje {\bf MSG1-}``El Estudiante [{\em Número de Boleta}] aun no puede inscribirse al seminario.''.
			\UCpaso[\UCactor] Oprime el botón \IUbutton{Aceptar}.
			\UCpaso[] Termina el caso de uso.
		\end{UCtrayectoriaA}
%------------------------------------- TERMINA caso de uso para dar de baja sucursal

%------------------------------------- COMIENZA caso de uso para ver sucursales en mapa
\begin{UseCase}{CU120}{Ruta en mapa interactivo}{
		Se despliega en la pantalla del actor un mapa en el cual podrá consultar la distancia que hay entre dos sucursales.
	}
		\UCitem{Versión}{0.1}
		\UCitem{Actor}{Gerente de operación del negocio, gerente de sucursal}
		\UCitem{Propósito}{Al saber la distancia y como llegar de una forma rápida de una sucursal a otra el gerente de sucursal sabrá si es coveniente o no mandar a un instructor a impartir una clase en alguna sucursal cercana.}
		\UCitem{Entradas}{Seleccion de la sucursal mostrada en pantalla}
		\UCitem{Origen}{Teclado}
		\UCitem{Salidas}{Se muestra el trazado de la  ruta  sobre el mapa incrustado en la pantalla}
		\UCitem{Destino}{Pantalla del actor}
		\UCitem{Precondiciones}{Tener alguna sucursal con falta de instructores. Que se necesiten mas instructores en alguna sucursal o por cambio de sucursal del personal}
		\UCitem{Postcondiciones}{Un intructor añadido a la sucursal}
		\UCitem{Errores}{}
		\UCitem{Tipo}{Caso de uso primario}
		\UCitem{Observaciones}{}
		\UCitem{Autor}{Fernández Quiñones Isaac.}
	\end{UseCase}

	\begin{UCtrayectoria}{Principal}
		\UCpaso[\UCactor] Ingresa a la plataforma web e ingresa su usuario y contraseña mediante la \IUref{IU23}{Pantalla de Control de Acceso}\label{CU17Login} para entrar en el sistema.
		\UCpaso Válida que el actor se encuentre dado de alta en el sistema. Se utiliza la regla \BRref{BR117}{Determinar si el usuario tiene acceso al sistema.} \Trayref{A}.
		\UCpaso Despliega la \IUref{IU32}{Pantalla de registro de nueva sucursal} con los campos necesarios para registrar en el sistema la nueva sucursal.
		\UCpaso[\UCactor] Llena todos los campos del formulario y envia el formulario. \Trayref{B}\label{CU17SeleccionarSeminario}.
		\UCpaso verifica los datos introducidos por el \UCactor. \BRref{BR118}{Determinar si los datos de los campos de un formulario son del tipo adecuado} \Trayref{C}.
		\UCpaso Almacena los datos en la base de datos.
		\UCpaso Muestra el \IUref{UI88}{Mensaje de registro exitoso}. 
		\UCpaso Pregunta al estudiante si desea imprimir un comprobante de la inscripción.		
	\end{UCtrayectoria}
		
		\begin{UCtrayectoriaA}{A}{El Estudiante no puede inscribir un Seminario}
			\UCpaso Muestra el Mensaje {\bf MSG1-}``El Estudiante [{\em Número de Boleta}] aun no puede inscribirse al seminario.''.
			\UCpaso[\UCactor] Oprime el botón \IUbutton{Aceptar}.
			\UCpaso[] Termina el caso de uso.
		\end{UCtrayectoriaA}





%PLANTILLA
%--------------------------------------
\begin{UseCase}{ID}{Nombre}{
		Descripcion
	}
		\UCitem{Versión}{0.1}
		\UCitem{Actor}{}
		\UCitem{Propósito}{}
		\UCitem{Entradas}{}
		\UCitem{Origen}{}
		\UCitem{Salidas}{}
		\UCitem{Destino}{}
		\UCitem{Precondiciones}{}
		\UCitem{Postcondiciones}{}
		\UCitem{Errores}{ }
		\UCitem{Tipo}{Caso de uso primario}
		\UCitem{Observaciones}{}
		\UCitem{Autor}{Fernández Quiñones Isaac.}
	\end{UseCase}

	\begin{UCtrayectoria}{Principal}
		\UCpaso[\UCactor] Ingresa a la plataforma web e ingresa su usuario y contraseña mediante la \IUref{IU23}{Pantalla de Control de Acceso}\label{CU17Login} para entrar en el sistema.
		\UCpaso Válida que el actor se encuentre dado de alta en el sistema. Se utiliza la regla \BRref{BR117}{Determinar si el usuario tiene acceso al sistema.} \Trayref{A}.
		\UCpaso Despliega la \IUref{IU32}{Pantalla de visualización de datos} para  consultar las sucursales registradas y un mapa en el cual podra ver la ubicación de cada sucursal y saber la distancia que  hay entre ellas.
		\UCpaso[\UCactor] Selecciona las sucursales \Trayref{B}\label{CU17SeleccionarSeminario}.
		\UCpaso verifica los datos introducidos por el \UCactor. \BRref{BR118}{Determinar si los datos de los campos de un formulario son del tipo adecuado} \Trayref{C}.
		\UCpaso Almacena los datos en la base de datos.
		\UCpaso Muestra el \IUref{UI88}{Mensaje de registro exitoso}. 
		\UCpaso Pregunta al estudiante si desea imprimir un comprobante de la inscripción.		
	\end{UCtrayectoria}
		
		\begin{UCtrayectoriaA}{A}{El Estudiante no puede inscribir un Seminario}
			\UCpaso Muestra el Mensaje {\bf MSG1-}``El Estudiante [{\em Número de Boleta}] aun no puede inscribirse al seminario.''.
			\UCpaso[\UCactor] Oprime el botón \IUbutton{Aceptar}.
			\UCpaso[] Termina el caso de uso.
		\end{UCtrayectoriaA}
