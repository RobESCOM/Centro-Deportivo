% \IUref{IUAdmPS}{Administrar Planta de Selección}
% \IUref{IUModPS}{Modificar Planta de Selección}
% \IUref{IUEliPS}{Eliminar Planta de Selección}

% 


% Copie este bloque por cada caso de uso:
%-------------------------------------- COMIENZA descripción del caso de uso.

%\begin{UseCase}[archivo de imágen]{UCX}{Nombre del Caso de uso}{

%------------------------------------- COMIENZA caso de uso para ver sucursales en mapa
\begin{UseCase}{CU14}{Ubicar mapa geograficamente}{
		Se muestra en la pantalla de ubicació un mapa, en el cual se podrán ubicar las diferentes sucursales establecidas en distintos estados de la república Mexicana.
	}
		\UCitem{Versión}{0.1}
		\UCitem{Actor}{Usuario}
		\UCitem{Propósito}{El usuario podrá localizar y trazar la mejor ruta para poder llegar a su sucursal de preferencia.}
		\UCitem{Entradas}{Selección de la sucursal mostrada en pantalla}
		\UCitem{Origen}{Teclado}
		\UCitem{Salidas}{Se muestra la ubicación de la sucursal en el mapa.}
		\UCitem{Destino}{Pantalla del actor}
		\UCitem{Precondiciones}{ninguna}
		\UCitem{Postcondiciones}{ninguna}
		\UCitem{Errores}{}
		\UCitem{Tipo}{Caso de uso primario}
		\UCitem{Observaciones}{}
		\UCitem{Autor}{Fernández Quiñones Isaac.}
		\UCitem{Revisó}{}
	\end{UseCase}

	\begin{UCtrayectoria}{Principal}
		\UCpaso[\UCactor] Ingresa a la plataforma web.
		\UCpaso El sistema muestra la \IUref{IU37}{Página de inicio.}
		\UCpaso[\UCactor] Selecciona la opción Ubicanos.
		\UCpaso Despliega la \IUref{IU33}{Pantalla de visualización de sucursal en mapa.}
		\UCpaso[\UCactor] Selecciona el estado en el que vive.
		\UCpaso Muestra la sucursal principal del estado seleccionado por el usuario.
		\UCpaso[\UCactor] Selecciona la sucursal mas cercana a su hogar.
		\UCpaso Pone una marca sobre la sucursal seleccionada por el [\UCactor].
	\end{UCtrayectoria}

%%PLANTILLA
%%--------------------------------------
%\begin{UseCase}{ID}{Nombre}{
%		Descripcion
%	}
%		\UCitem{Versión}{0.1}
%		\UCitem{Actor}{}
%		\UCitem{Propósito}{}
%		\UCitem{Entradas}{}
%		\UCitem{Origen}{}
%		\UCitem{Salidas}{}
%		\UCitem{Destino}{}
%		\UCitem{Precondiciones}{}
%		\UCitem{Postcondiciones}{}
%		\UCitem{Errores}{ }
%		\UCitem{Tipo}{Caso de uso primario}
%		\UCitem{Observaciones}{}
%		\UCitem{Autor}{Fernández Quiñones Isaac.}
%	\end{UseCase}
%
%	\begin{UCtrayectoria}{Principal}
%		\UCpaso[\UCactor] Ingresa a la plataforma web e ingresa su usuario y contraseña mediante la \IUref{IU23}{Pantalla de Control de Acceso}\label{CU17Login} para entrar en el sistema.
%		\UCpaso Válida que el actor se encuentre dado de alta en el sistema. Se utiliza la regla \BRref{BR117}{Determinar si el usuario tiene acceso al sistema.} \Trayref{A}.
%		\UCpaso Despliega la \IUref{IU32}{Pantalla de visualización de datos} para  consultar las sucursales registradas y un mapa en el cual podra ver la ubicación de cada sucursal y saber la distancia que  hay entre ellas.
%		\UCpaso[\UCactor] Selecciona las sucursales \Trayref{B}\label{CU17SeleccionarSeminario}.
%		\UCpaso verifica los datos introducidos por el \UCactor. \BRref{BR118}{Determinar si los datos de los campos de un formulario son del tipo adecuado} \Trayref{C}.
%		\UCpaso Almacena los datos en la base de datos.
%		\UCpaso Muestra el \IUref{UI88}{Mensaje de registro exitoso}. 
%		\UCpaso Pregunta al estudiante si desea imprimir un comprobante de la inscripción.		
%	\end{UCtrayectoria}
%		
%		\begin{UCtrayectoriaA}{A}{El Estudiante no puede inscribir un Seminario}
%			\UCpaso Muestra el Mensaje {\bf MSG1-}``El Estudiante [{\em Número de Boleta}] aun no puede inscribirse al seminario.''.
%			\UCpaso[\UCactor] Oprime el botón \IUbutton{Aceptar}.
%			\UCpaso[] Termina el caso de uso.
%		\end{UCtrayectoriaA}
