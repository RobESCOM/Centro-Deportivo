\subsection{IU18 Pantalla de menú de ocpiones de la sección áreas.}

\subsubsection{Objetivo}
	El gerente podrá seleccionar una opcion del menú areas.

\subsubsection{Diseño}
	Esta pantalla se encuentra en la seccion de opciones del menu áreas de la sección principal. 

\IUfig[.5]{gui/menuOpciones}{IU18}{Pantalla de Menú de opciones.}

\subsubsection{Salidas}

	Ninguna.

\subsubsection{Entradas}
El gerente podrá escoger alguna opción del menú con tan solo dar un click.

\subsubsection{Comandos}
\begin{itemize}
	\item \IUbutton{Registrar áreas}: Muestra un formulario para registrar áreas en la sucursal.
	\item \IUbutton{Consultar áreas}: Muestra las áreas registradas en la sucursal.
	\item \IUbutton{Eliminar áreas}: Muestra una lista desplegable donde el gerente puede eliminar las áreas registradas.
	\item \IUbutton{Actualizar áreas}: Muestra un formulario en donde se pueden modificar los datos de las áreas.
\end{itemize}

\subsubsection{Mensajes}
	\begin{Citemize}
		\item {\bf MSG0} Ningun mensaje de error.
	\end{Citemize}

%%%%%%%%%%%%%%%%%%%%%%%%%%%%%%%%%%%%%%%%%%%%%%%%%%%%%%%%%%%%%%%%%%%%%%%%%%%%%%%%%	
%%%%%%%%%%%%%%%%%%%%%%%%%%%%%%%%%%%%%%%%%%%%%%%%%%%%%%%%%%%%%%%%%%%%%%%%%%%%%%%%%
\subsection{IU19 Pantalla de registro de área}

\subsubsection{Objetivo}
	El gerente podrá registrar nuevas áreas en la sucursal.

\subsubsection{Diseño}
	Esta pantalla en la opción registrar áreas que se encuentra en la seccion de opciones del menu áreas de la sección principal.

\IUfig[.5]{gui/registroArea}{IU19}{Pantalla de registo de nueva área.}

\subsubsection{Salidas}
Mensaje de área registrada exitosamente.

\subsubsection{Entradas}
Nombre de área, Tipo de área, Largo, Ancho, Capacidad, Responsable y Descripción.

\subsubsection{Comandos}
\begin{itemize}
	\item \IUbutton{Registrar área}: Es la acción para dar de alta una nueva área.
	\item \IUbutton{Limpiar}: Limpia el formulario, deja los campos en blanco.
\end{itemize}

\subsubsection{Mensajes}
	\begin{Citemize}
		\item {\bf MSG1} Campo nombre: Debe contener letras y espacios en blanco.
	\end{Citemize}
%%%%%%%%%%%%%%%%%%%%%%%%%%%%%%%%%%%%%%%%%%%%%%%%%%%%%%%%%%%%%%%%%%%%%%%%%%%%%%%%%	
%%%%%%%%%%%%%%%%%%%%%%%%%%%%%%%%%%%%%%%%%%%%%%%%%%%%%%%%%%%%%%%%%%%%%%%%%%%%%%%%%
\subsection{IU20 Pantalla de Eliminar áreas}

\subsubsection{Objetivo}
	El gerente podrá dar de baja alguna área que se encuentre registrada.

\subsubsection{Diseño}
	Esta pantalla se encuentra en la sección de áreas del menú principal.

\IUfig[.5]{gui/EliminaAreas}{IU20}{Pantalla de eliminar áreas.}

\subsubsection{Salidas}

	Mensaje de registro eliminado correctamete.

\subsubsection{Entradas}
Entrada directamente sobre el mouse, en la opción eliminar de la lista desplegable de las áreas.

\subsubsection{Comandos}
\begin{itemize}
	\item \IUbutton{Eliminar}: Esta opción permite eliminar un área registrada.
	\item \IUbutton{Regresar al menú principal}: Esta opción permite regresar al menú principal.
\end{itemize}

\subsubsection{Mensajes}
	\begin{Citemize}
		\item {\bf MSG5} Ningun mensaje de error.
	\end{Citemize}
%%%%%%%%%%%%%%%%%%%%%%%%%%%%%%%%%%%%%%%%%%%%%%%%%%%%%%%%%%%%%%%%%%%%%%%%%%%%%%%%%	
%%%%%%%%%%%%%%%%%%%%%%%%%%%%%%%%%%%%%%%%%%%%%%%%%%%%%%%%%%%%%%%%%%%%%%%%%%%%%%%%%

\subsection{IU22 Pantalla de Actualización de áreas}

\subsubsection{Objetivo}
	El gerente podrá actualizar una área qeu se encuentre registrada a fin de modificar su contenido.

\subsubsection{Diseño}
	Esta pantalla se encuentra en la sección de áreas del menú principal.

\IUfig[.5]{gui/registroArea2}{IU22}{Pantalla de eliminar áreas.}

\subsubsection{Salidas}

	Mensaje de datos actualizados correctamente.

\subsubsection{Entradas}
En caso de ser necesario el gerente podrá actualizar los datos: Nombre de área, Tipo de área, Largo, Ancho, Capacidad, Responsable y Descripción.

\subsubsection{Comandos}
\begin{itemize}
	\item \IUbutton{Registrar área}: Es la acción para dar de alta una nueva área.
	\item \IUbutton{Limpiar}: Limpia el formulario, deja los campos en blanco.
\end{itemize}

\subsubsection{Mensajes}
	\begin{Citemize}
		\item {\bf MSG5} Ningun mensaje de error.
	\end{Citemize}

%%%%%%%%%%%%%%%%%%%%%%%%%%%%%%%%%%%%%%%%%%%%%%%%%%%%%%%%%%%%%%%%%%%%%%%%%%%%%%%%%	
%%%%%%%%%%%%%%%%%%%%%%%%%%%%%%%%%%%%%%%%%%%%%%%%%%%%%%%%%%%%%%%%%%%%%%%%%%%%%%%%%
\subsection{IU23 Pantalla de Control de Acceso}

\subsubsection{Objetivo}
	Controlar el acceso al sistema mediante una contraseña a fin de que cada usuario acceda solo a las operaciones permitidas para su perfil.

\subsubsection{Diseño}
	Esta pantalla aparece al iniciar el sistema. Para ingresar al mismo se debe escribir el Número de Boleta del estudiante y la contraseña de acceso. 

\IUfig[.5]{gui/Login}{IU23}{Pantalla de Control de Acceso.}

\subsubsection{Salidas}

	Ninguna.

\subsubsection{Entradas}
Número de Boleta y Contraseña del Estudiante.

\subsubsection{Comandos}
\begin{itemize}
	\item \IUbutton{Entrar}: Verifica que el Estudiante se encuentre registrado y la contraseña sea la correcta. Si la verificación es correcta, se muestra la \IUref{IU32}{Pantalla de Selección de Seminario}.
	\item \IUbutton{Ayuda}: Muestra la ayuda de esta pantalla \IUref{IU50}{Pantalla de Ayuda}.
\end{itemize}

\subsubsection{Mensajes}
	\begin{Citemize}
		\item {\bf MSG5} Error al verificar los datos de acceso, vuelva a intentarlo.
	\end{Citemize}
	
%%%%%%%%%%%%%%%%%%%%%%%%%%%%%%%%%%%%%%%%%%%%%%%%%%%%%%%%%%%%%%%%%%%%%%%%%%%%%%%%%	
%%%%%%%%%%%%%%%%%%%%%%%%%%%%%%%%%%%%%%%%%%%%%%%%%%%%%%%%%%%%%%%%%%%%%%%%%%%%%%%%%

\subsection{IU33 Pantalla de visualización de sucursal en mapa.}

\subsubsection{Objetivo}
	Mostrar un mapa interactivo, en el cual se tiene dos selectores. Uno funciona para ubicar las sucursales en diferentes estados de la república Mexicana. El segundo para localizar la sucursal dentro del estado seleccionado.

\subsubsection{Diseño}
	Esta pantalla se despliega cuando cualquier persona da click sobre la opción ubicación.

\IUfig[.5]{gui/mapaU}{IU33}{Pantalla de visualización de sucursal en mapa.}

\subsubsection{Salidas}

	Ninguna.

\subsubsection{Entradas}
Nombre de las sucursales

\subsubsection{Comandos}
\begin{itemize}
	\item \IUbutton{Entrar}: Muestra un mapa interactivo \IUref{IU33}{Pantalla de visualización de sucursal en mapa}.
\end{itemize}

\subsubsection{Mensajes}
	\begin{Citemize}
		\item {\bf MSG5} Ninguno.
	\end{Citemize}
	
%%%%%%%%%%%%%%%%%%%%%%%%%%%%%%%%%%%%%%%%%%%%%%%%%%%%%%%%%%%%%%%%%%%%%%%%%%%%%%%%%	
%%%%%%%%%%%%%%%%%%%%%%%%%%%%%%%%%%%%%%%%%%%%%%%%%%%%%%%%%%%%%%%%%%%%%%%%%%%%%%%%%
	

\subsection{IU32 Pantalla de formulario para registro de nueva sucursal.}

\subsubsection{Objetivo}
	Mostrarlos campos necesarios para dar de alta en el sistema una nueva sucursal.

\subsubsection{Diseño}
	Esta pantalla aparece cuando el gerente de operación se registro en el sistema.

\IUfig[.5]{gui/registro}{IU32}{Pantalla de formulario para registro de nueva sucursal}

\subsubsection{Salidas}

	Ninguna.

\subsubsection{Entradas}
	Nombre de la sucursal, fecha de inauguración, estado, teléfonos, correos electrónicos, dirección de la sucursal, áreas, que tendrá la nueva sucursal, datos del gerente de la sucursal y una descripción de la sucursal.

\subsubsection{Comandos}
\begin{itemize}
	\item \IUbutton{Entrar}: Muestra los campos que son necesarios se llenen para dar de alta la sucursal \IUref{IU22}{Pantalla de registro de nueva sucursal}.
	\item \IUbutton{Ayuda}: Muestra la ayuda de esta pantalla \IUref{IU50}{Pantalla de Ayuda}.
\end{itemize}

\subsubsection{Mensajes}
	\begin{Citemize}
		\item {\bf MSG5} Ninguno.
	\end{Citemize}

%%%%%%%%%%%%%%%%%%%%%%%%%%%%%%%%%%%%%%%%%%%%%%%%%%%%%%%%%%%%%%%%%%%%%%%%%%%%%%%%%	
%%%%%%%%%%%%%%%%%%%%%%%%%%%%%%%%%%%%%%%%%%%%%%%%%%%%%%%%%%%%%%%%%%%%%%%%%%%%%%%%%
	
\subsection{IU99 Pantalla dar de baja sucursales}

\subsubsection{Objetivo}
	Mostrar en forma tabular las sucursales existentes. Así se podrá seleccionar cual sucursal se quiere eliminar

\subsubsection{Diseño}
	Esta pantalla aparece cuando el gerente de operación se registro en el sistema.

\IUfig[.5]{gui/baja}{IU99}{Pantalla dar de baja sucursales}

\subsubsection{Salidas}

	Ninguna.

\subsubsection{Entradas}
	Ninguna.

\subsubsection{Comandos}
\begin{itemize}
	\item \IUbutton{Entrar}: Muestra en forma tablar las sucursales registradas en el sistema. sucursal \IUref{IU99}{Pantalla dar de baja sucursales}.
	\item \IUbutton{Ayuda}: Muestra la ayuda de esta pantalla \IUref{IU50}{Pantalla de Ayuda}.
\end{itemize}

\subsubsection{Mensajes}
	\begin{Citemize}
		\item {\bf MSG5} Ninguno.
	\end{Citemize}
	
%%%%%%%%%%%%%%%%%%%%%%%%%%%%%%%%%%%%%%%%%%%%%%%%%%%%%%%%%%%%%%%%%%%%%%%%%%%%%%%%%	
%%%%%%%%%%%%%%%%%%%%%%%%%%%%%%%%%%%%%%%%%%%%%%%%%%%%%%%%%%%%%%%%%%%%%%%%%%%%%%%%%

\subsection{IU97 Pantalla para consultar sucursales registradas}

\subsubsection{Objetivo}
	Mostrar en forma tabular las sucursales existentes. Así se podrá seleccionar cual sucursal se quiere actualizar

\subsubsection{Diseño}
	Esta pantalla aparece cuando el gerente de operación se registro en el sistema.

\IUfig[.5]{gui/consulta}{IU97}{Pantalla para consultar sucursales registradas}

\subsubsection{Salidas}

	Ninguna.

\subsubsection{Entradas}
	Ninguna.

\subsubsection{Comandos}
\begin{itemize}
	\item \IUbutton{Entrar}: Muestra en forma tablar las sucursales registradas en el sistema. sucursal \IUref{IU97}{Pantalla para consultar sucursales registradas}.
	\item \IUbutton{Ayuda}: Muestra la ayuda de esta pantalla \IUref{IU50}{Pantalla de Ayuda}.
\end{itemize}

\subsubsection{Mensajes}
	\begin{Citemize}
		\item {\bf MSG5} Ninguno.
	\end{Citemize}

%%%%%%%%%%%%%%%%%%%%%%%%%%%%%%%%%%%%%%%%%%%%%%%%%%%%%%%%%%%%%%%%%%%%%%%%%%%%%%%%%	
%%%%%%%%%%%%%%%%%%%%%%%%%%%%%%%%%%%%%%%%%%%%%%%%%%%%%%%%%%%%%%%%%%%%%%%%%%%%%%%%%	

\subsection{IU34 Añadir otro campo}

\subsubsection{Objetivo}
	Añadir al formulario un nuevo campo de texto. Éste se llenará con el valor alfanumérico en el caso que se quiera añadir un correo o se llenará con un número si se quiere agregar un número teléfonico.

\subsubsection{Diseño}
	Se muestra el nuevo campo de texto en el formulario.

\IUfig[.5]{gui/NTelefon}{IU34}{Añadir otro campo}

\subsubsection{Salidas}

	Ninguna.

\subsubsection{Entradas}
	El valor alfanumérico con el cual se quiera llenar el nuevo campo. Este valor pede ser unicamente numérico si se pide agregar un teléfono distinto.

\subsubsection{Comandos}
\begin{itemize}
	\item \IUbutton{Entrar}: Muestra el nuevo campo \IUref{IU34}{Añadir otro campo}.
	\item \IUbutton{Ayuda}: Muestra la ayuda de esta pantalla \IUref{IU50}{Pantalla de Ayuda}.
\end{itemize}

\subsubsection{Mensajes}
	\begin{Citemize}
		\item {\bf MSG5} Pide la confirmación para añadir otro campo.
	\end{Citemize}
	
%%%%%%%%%%%%%%%%%%%%%%%%%%%%%%%%%%%%%%%%%%%%%%%%%%%%%%%%%%%%%%%%%%%%%%%%%%%%%%%%%	
%%%%%%%%%%%%%%%%%%%%%%%%%%%%%%%%%%%%%%%%%%%%%%%%%%%%%%%%%%%%%%%%%%%%%%%%%%%%%%%%%

\subsection{IU35 Añadir otro email}

\subsubsection{Objetivo}
	Añadir al formulario un nuevo campo de texto. Éste se llenará con el valor alfanumérico correspondiente a una dirección de correo electrónico válida según la siguiente expresión regular: %/^[a-z]([\w\.]*)@[a-z]([\w\.]*)\.[a-z]{2,3}/.

\subsubsection{Diseño}
	Se muestra el nuevo campo de texto en el formulario.

\IUfig[.5]{gui/NMail}{IU35}{Añadir otro email}

\subsubsection{Salidas}

	Ninguna.

\subsubsection{Entradas}
	El valor alfanumérico con el cual se quiera llenar el nuevo campo. 

\subsubsection{Comandos}
\begin{itemize}
	\item \IUbutton{Entrar}: Muestra el nuevo campo \IUref{IU35}{Añadir otro email}.
	\item \IUbutton{Ayuda}: Muestra la ayuda de esta pantalla \IUref{IU50}{Pantalla de Ayuda}.
\end{itemize}

\subsubsection{Mensajes}
	\begin{Citemize}
		\item {\bf MSG5} Pide la confirmación para añadir otro campo.
	\end{Citemize}

%%%%%%%%%%%%%%%%%%%%%%%%%%%%%%%%%%%%%%%%%%%%%%%%%%%%%%%%%%%%%%%%%%%%%%%%%%%%%%%%%	
%%%%%%%%%%%%%%%%%%%%%%%%%%%%%%%%%%%%%%%%%%%%%%%%%%%%%%%%%%%%%%%%%%%%%%%%%%%%%%%%%

\subsection{IU37 Página de inicio}

\subsubsection{Objetivo}
	Mostrar una pantalla de bienvenida al usuario, además de los tipos de membresía con los que cuenta el centro deportivo San Pancho.

\subsubsection{Diseño}
	Muestra las opciones disponible para el usuario.

\IUfig[.5]{gui/inicio}{IU37}{Página de inicio}

\subsubsection{Salidas}

	Ninguna.

\subsubsection{Entradas}
	Ninguna.

\subsubsection{Comandos}
\begin{itemize}
	\item \IUbutton{Entrar}: Muestra el nuevo campo \IUref{IU37}{Página de inicio}.
\end{itemize}

\subsubsection{Mensajes}
	\begin{Citemize}
		\item Ninguno.
	\end{Citemize}