% \IUref{IUAdmPS}{Administrar Planta de Selección}
% \IUref{IUModPS}{Modificar Planta de Selección}
% \IUref{IUEliPS}{Eliminar Planta de Selección}

%-------------------------------------- TERMINA descripción del caso de uso.

%\begin{UseCase}[archivo de imágen]{UCX}{Nombre del Caso de uso}{
	\begin{UseCase}{CU10.0}{Desactivar área}{
		Esta sección es diferente a la de eliminar áreas, ya que el propósito de éste módulo consiste en desactivar temporalmente un área que no tenga actividad por el momento, no será eliminada permanentemente si no que ya no aparecerá en las áreas registradas, hasta que vuelva a ser activada.
	}
		\UCitem{Versión}{1.0}
		\UCitem{Actor}{Gerente de Sucursal.}
		\UCitem{Propósito}{Que se pueda cambiar el estado activo de un área en caso de que existan modificaciones en el uso.}
		\UCitem{Entradas}{Nombre del Área, lista desplegable con las áreas.}
		\UCitem{Origen}{Desde el teclado o mouse.}

		\UCitem{Salidas}{{\bf MSG19-}``El área ha sido desactivada.''}

		\UCitem{Destino}{El estado del área se verá reflejado en el área de consultas, ya que no aparecerá dentro de esa sección.
Precondiciones: Que el área se encuentre previamente registrada.}
		\UCitem{Precondiciones}{Que el área se encuentre registrada previamente.El administrador es el único usuario que puede dar de baja un área.}
		\UCitem{Postcondiciones}{El área registrada se verá reflejada en la sección de consultas.}

		\UCitem{Errores}{
{\bf E7:} ``No se tiene ningún registro.'' -- El sistema muestra el Mensaje {\bf MSG7-}``No existe ningun registro.''}

		\UCitem{Tipo}{Caso de uso primario.}
		\UCitem{Observaciones}{}
		\UCitem{Autor}{Francisco García Enríquez.}
		\UCitem{Revisor}{Martin Carrillo.}
	\end{UseCase}

\begin{UCtrayectoria}{Principal}
		\UCpaso[\UCactor] Solicita el ingreso al apartado de clientes seleccionando la opción ``Clientes'' de la \IUref{IU6}{Pantalla de perfil de empleado}.
		\UCpaso Toma la sesión del gerente de sucursal.
		\UCpaso[\UCactor] Selecciona del menú principal la opción Áreas.
		\UCpaso Muestra las opciones que el gerente pueda realizar: Registrar Áreas, Consultar Áreas, Eliminar Áreas, Desactivar Áreas y Actualizar Áreas.
		\UCpaso[\UCactor] Selecciona la opción desactivar áreas.
		\UCpaso Carga en una lista desplegable las áreas disponible registradas. \Trayref{A}.
		\UCpaso[\UCactor] Selecciona de la lista delegable el área que desea desactivar.
		\UCpaso[\UCactor] Confirma la operación y presiona el botón desactivar.
		\UCpaso Inhabilita en área y le muestra el mensaje {\bf MSG19-}``El área ha sido desactivada.''
		\UCpaso Le muestra una opción para regresar al menú de opciones.
	\end{UCtrayectoria}

\begin{UCtrayectoriaA}{B}{No existe ningun dato registrado.}
			\UCpaso[\UCactor] Muestra el mensaje {\bf MS7-}``No existe ningun registro''
			\UCpaso[\UCactor] Finaliza su operación dentro del área de consulta. 
			\UCpaso[\UCactor] Puede regresar al menú de opciones de la sección \IUref{IU18}{Pantalla de menu de opciones áreas} mediante el botón de menú de inicio.
		\end{UCtrayectoriaA}

%-------------------------------------- TERMINA descripción del caso de uso.